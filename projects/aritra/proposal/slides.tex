% Copyright (c) 2022 by Lars Spreng
% This work is licensed under the Creative Commons Attribution 4.0 International License. 
% To view a copy of this license, visit http://creativecommons.org/licenses/by/4.0/ or send a letter to Creative Commons, PO Box 1866, Mountain View, CA 94042, USA.

%~~~~~~~~~~~~~~~~~~~~~~~~~~~~~~~~~~~~~~~~~~~~~~~~~~~~~~~~~~~~~~~~~~~~~~~~~~~~~~
% You can add your packages and commands to the loadslides.tex file. 
% The files in the folder "styles" can be modified to change the layout and design of your slides.
% I have included examples on how to use the template below. 
% Some of it these examples are taken from the Metropolis template.
%~~~~~~~~~~~~~~~~~~~~~~~~~~~~~~~~~~~~~~~~~~~~~~~~~~~~~~~~~~~~~~~~~~~~~~~~~~~~~~


\documentclass[
11pt,notheorems,hyperref={pdfauthor=Aritra Mukhopadhyay}
]{beamer}

\input{loadslides.tex} % Loads packages and some defined commands

\title[
% Text entered here will appear in the bottom middle
Project proposal
]{Presentation Title}

\subtitle{Presentation Subtitle}

\author[
% Text entered here will appear in the bottom left corner
Aritra Mukhopadhyay
]{
    John Doe 
}

\institute{
    Author Affiliation, \\
    University of Author}
\date{\today}

\begin{document}

% Generate title page
{
\setbeamertemplate{footline}{} 
% \begin{frame}
%   \titlepage
% \end{frame}
}
\addtocounter{framenumber}{-1}

% You can declare different parts as a parentof sections
% \begin{frame}{Part I: Demo Presentation Part}
%     \tableofcontents[part=1]
% \end{frame}
% \begin{frame}{Part II: Demo Presentation Part 2}
%     \tableofcontents[part=2]
% \end{frame}

% \makepart{Demo Part}

\begin{frame}{\textbf{Neural Networks at a Fraction:} Table Structure Recognition
}{Aritra Mukhopadhyay}
\alert{\textbf{Objective}}: To make quaternion versions of the Table Transformer (TATR) model and deploy in a low powered mobile device with limited memory and computational power.

\alert{\textbf{Dataset:}} PubTables-1M, FinTabNet, ICDAR 2013

\alert{\textbf{Baseline models:}} Table Transformer (TATR) model



% - [End-to-End Object Detection with Transformers (2020)](https://scontent-ccu1-1.xx.fbcdn.net/v/t39.2365-6/154305880_816694605586461_2873294970659239190_n.pdf?_nc_cat=108&ccb=1-7&_nc_sid=3c67a6&_nc_ohc=DkHwrRBBc_cAX-eMpS7&_nc_ht=scontent-ccu1-1.xx&oh=00_AfCVQywPZJ9qTTMxk3f6OzsXUEBE9ASe4JMTKI1zE1gCqQ&oe=64F07EC3)
%   from *Facebook AI Research (Nicolas Carion, Francisco Massa et al.)*
% - [GriTS: Grid table similarity metric for table structure recognition](https://arxiv.org/pdf/2203.12555)
%   from *Microsoft Research (Brandon Smock, Rohith Pesala, Robin Abraham)* (2022)
% - [Aligning benchmark datasets for table structure recognition](https://arxiv.org/pdf/2303.00716)
%   from *Microsoft Research (Brandon Smock, Rohith Pesala, Robin Abraham)* (2023)



\alert{\textbf{Relevant Papers}}:
\begin{itemize}
    \item \href{https://scontent-ccu1-1.xx.fbcdn.net/v/t39.2365-6/154305880_816694605586461_2873294970659239190_n.pdf?_nc_cat=108&ccb=1-7&_nc_sid=3c67a6&_nc_ohc=DkHwrRBBc_cAX-eMpS7&_nc_ht=scontent-ccu1-1.xx&oh=00_AfCVQywPZJ9qTTMxk3f6OzsXUEBE9ASe4JMTKI1zE1gCqQ&oe=64F07EC3}{End-to-End Object Detection with Transformers (2020)}\\
    from \textit{Facebook AI Research (Nicolas Carion, Francisco Massa et al.)}

    \item \href{https://arxiv.org/pdf/2103.12555}{GriTS: Grid table similarity metric for table structure recognition (2022)}\\
    from \textit{Microsoft Research (Brandon Smock, Rohith Pesala, Robin Abraham)}

    \item \href{https://arxiv.org/pdf/2303.00716}{Aligning benchmark datasets for table structure recognition (2023)}\\
    from \textit{Microsoft Research (Brandon Smock, Rohith Pesala, Robin Abraham)}
\end{itemize}


% \alert{\textbf{Work Division}}
% \begin{itemize}
%     \item \textbf{Aritra:} Reading papers,  Experimenting with models, tackling coding and data hurdles, and analyzing results.
%     \item \textbf{Adhil:} Reading papers, documenting the project. tuning and experimenting with hyperparameters, and analyzing the results. 

% \end{itemize}
     
\end{frame}


\begin{frame}{}
    \alert{\textbf{Midway Plans:}}
    \begin{itemize}
        \item Understand the GriTS metrics
        \item Learn about transformers
        \item Understand the DE:TR model and pipeline
        \item Understand the TATR model and dataset
        \item LTH on pretrained TATR on the FinTabNet dataset
        \item Compare Finetuned pruned model vs pruned finetuned model
    \end{itemize}

    \alert{\textbf{Further Plans:}}
    \begin{itemize}
        \item Make quaternion version of the TATR model
        \item LTH on Quaternion TATR model
        \item Production
    \end{itemize}

    \alert{\textbf{Expected Results:}}

    Microsoft Lens App will have a better Table Structure Recognition model that will be able to run on low powered mobile devices with limited memory and computational power.

\end{frame}

\end{document}